%%%%%%%%%%%%%%%%%%%%%%%%%%%%%%%%%%%%%%%
% Deedy - One Page Two Column Resume
% LaTeX Template
% Version 1 (30/4/2014)
%
% Original author:
% Debarghya Das (http://debarghyadas.com)
%
% Original repository:
% https://github.com/deedydas/Deedy-Resume
%
% IMPORTANT: THIS TEMPLATE NEEDS TO BE COMPILED WITH XeLaTeX
%
% This template uses several fonts not included with Windows/Linux by
% default. If you get compilation errors saying a font is missing, find the line
% on which the font is used and either change it to a font included with your
% operating system or comment the line out to use the default font.
% 
%%%%%%%%%%%%%%%%%%%%%%%%%%%%%%%%%%%%%%
% 
% TODO:
% 1. Integrate biber/bibtex for article citation under publications.
% 2. Figure out a smoother way for the document to flow onto the next page.
% 3. Add styling information for a "Projects/Hacks" section.
% 4. Add location/address information
% 5. Merge OpenFont and MacFonts as a single sty with options.
% 
%%%%%%%%%%%%%%%%%%%%%%%%%%%%%%%%%%%%%%
%
% CHANGELOG:
% v1.1:
% 1. Fixed several compilation bugs with \renewcommand
% 2. Got Open-source fonts (Windows/Linux support)
% 3. Added Last Updated
% 4. Move Title styling into .sty
% 5. Commented .sty file.
%
%%%%%%%%%%%%%%%%%%%%%%%%%%%%%%%%%%%%%%%
%
% Known Issues:
% 1. Overflows onto second page if any column's contents are more than the
% vertical limit
% 2. Hacky space on the first bullet point on the second column.
%
%%%%%%%%%%%%%%%%%%%%%%%%%%%%%%%%%%%%%%
\usepackage{hyperref}
\documentclass[]{deedy-resume-openfont}


\begin{document}


%%%%%%%%%%%%%%%%%%%%%%%%%%%%%%%%%%%%%%
%
%     LAST UPDATED DATE
%
%%%%%%%%%%%%%%%%%%%%%%%%%%%%%%%%%%%%%%
\lastupdated

%%%%%%%%%%%%%%%%%%%%%%%%%%%%%%%%%%%%%%
%
%     TITLE NAME
%
%%%%%%%%%%%%%%%%%%%%%%%%%%%%%%%%%%%%%%


\namesection{Shivam}{Patel}{ \urlstyle{same} \\
\href{ mail to:}{patelshivam99@gmail.com} 
}

%%%%%%%%%%%%%%%%%%%%%%%%%%%%%%%%%%%%%%
%
%     COLUMN ONE
%
%%%%%%%%%%%%%%%%%%%%%%%%%%%%%%%%%%%%%%

\begin{minipage}[t]{0.33\textwidth} 

%%%%%%%%%%%%%%%%%%%%%%%%%%%%%%%%%%%%%%
%     EDUCATION
%%%%%%%%%%%%%%%%%%%%%%%%%%%%%%%%%%%%%%

\section{Education} 

\subsection{Nirma University}
\descript{BTech in Computer Engineering}
\location{Expected May 2020 |Gujarat,India}
\sectionsep


%\descript{BS in Computer Science}
%\location{Expected May 2014 | Ithaca, NY}
%Conc. in Software Engineering \\
%College of Engineering \\
%Dean's List (All Semesters) \\
%\location{ Cum. GPA: 3.92 / 4.0 \\
%Major GPA: 3.94 / 4.0}
\sectionsep

\subsection{St. Xavier's Loyola Hall}
\location{Grad. May 2016|Ahmedabad, India}
\sectionsep

%%%%%%%%%%%%%%%%%%%%%%%%%%%%%%%%%%%%%%
%     LINKS
%%%%%%%%%%%%%%%%%%%%%%%%%%%%%%%%%%%%%%

\section{Links} 
Github:// \href{https://github.com/Shivamshaiv}{\custombold{Shivamshaiv}} \\
LinkedIn://  \href{https://www.linkedin.com/in/shivam-p-8b7b3585/}{\custombold{shivam-p-8b7b3585}} \\
Quora://  \href{https://www.quora.com//Shivam-Patel-32}{\custombold{Shivam-Patel-32}}
\sectionsep

%%%%%%%%%%%%%%%%%%%%%%%%%%%%%%%%%%%%%%
%     SKILLS
%%%%%%%%%%%%%%%%%%%%%%%%%%%%%%%%%%%%%%

\section{Skills}
\subsection{Programming}
\location{Over 10000 lines:}
 \textbullet{} Python \textbullet{} C \textbullet{} C++ \textbullet{} Java \\

\textbullet{} Mathematica \textbullet{} Matlab \textbullet{} Javascript\\
%Java \textbullet{}   Shell \textbullet{} JavaScript \textbullet{} Matlab \\
%OCaml \textbullet{} Python \textbullet{} Rails \textbullet{} \LaTeX\ \\ 
\location{Over 2000 lines:}
\textbullet{} CUDA \textbullet{} Hadoop \textbullet{} CSS \\ \textbullet{} PHP \textbullet{} Android \\
\location{Deep Learning:}
\textbullet{} Tensorflow \textbullet{} Keras \textbullet{} Caffe \\
\textbullet{} Pytorch \textbullet{} PySpark \textbullet{} TfLearn \\
\location{Operating systems:}
\textbullet{} Windows \textbullet{} Unix \textbullet{} Linux \\
\location{Familiar:}
AS3 \textbullet{} iOS \textbullet{} Assembly \textbullet{} MySQL\\
\location{Design tools:}
\textbullet{} Adobe Photoshop \textbullet{} SolidWorks \\
\textbullet{} Blender \textbullet{} CorelDraw


\sectionsep



%%%%%%%%%%%%%%%%%%%%%%%%%%%%%%%%%%%%%%
%     COURSEWORK
%%%%%%%%%%%%%%%%%%%%%%%%%%%%%%%%%%%%%%

\section{Coursework}
\begin{comment}
\subsection{MIT OCW}
\begin{tabular}{r11}
\underline{AGE} &\underline{COURSE}	 \\
8	  &Number theory\\
9	 &Single Variable Calculus\\   
9	  &Multivariable Calculus\\   
10    &Analysis I \& II \\
10    &Linear Algebra   \\
11    &Introduction to Graph Theory\\
  
\end{tabular}
\end{comment}


\begin{comment}


Open Source Software Engineering \\
Advanced Interactive Graphics \\
Compilers + Practicum \\
Cloud Computing \\
\end{comment}

\subsection{Online}
Introduction to MATLAB Programming\\
Practical Programming in C:MIT\\
Programming Paradigms:Stanford \\
Introduction to Algorithms: MIT \\
Cryptography I \& II :Stanford\\
Artificial Intelligence:MIT\\
{\footnotesize \textit{\textbf{(And over 120 others) }}} \\
\sectionsep


\subsection{Open Source Projects}
TensorFlow : Feature addition\\
Keras : Feature addition \\
OpenCV 3:Feature addition\\
DeepCell : Feature addition \\
tfRetinanet : Converted to tf.keras \\
GTText : Optimization \\
Matplotlib : Bug Fixes\\
Project Jupyter : Bug Fixes \\
Mozilla : Front End\\

{\footnotesize \textit{\textbf{(And so many more ...) }}} \\
\sectionsep


%%%%%%%%%%%%%%%%%%%%%%%%%%%%%%%%%%%%%%
%
%     COLUMN TWO
%
%%%%%%%%%%%%%%%%%%%%%%%%%%%%%%%%%%%%%%

\end{minipage} 
\hfill
\begin{minipage}[t]{0.66\textwidth} 

%%%%%%%%%%%%%%%%%%%%%%%%%%%%%%%%%%%%%%
%     EXPERIENCE
%%%%%%%%%%%%%%%%%%%%%%%%%%%%%%%%%%%%%%
\begin{comment}


\section{Experience}

\runsubsection{Coursera}
\descript{| KPCB Fellow + Software Engineering Intern }
\location{Expected June 2014 – Sep 2014 | Mountain View, CA}
\vspace{\topsep} % Hacky fix for awkward extra vertical space
\begin{tightemize}\item 52 out of 2500 applicants chosen to be a KPCB Fellow 2014.
\end{tightemize}
\sectionsep

\runsubsection{Google}
\descript{| Software Engineering Intern }
\location{May 2013 – Aug 2013 | Mountain View, CA}
\begin{tightemize}
\item Worked on the YouTube Captions team in primarily vanilla Javascript and Python to plan, design and develop the full stack implementation of a new framework to add and edit Automatic Speech Recognition captions.\item Created a backbone.js-like framework for the Captions editor.\item All code was reviewed, perfected, and pushed to production.\end{tightemize}
\sectionsep

\runsubsection{Phabricator}
\descript{| Open Source Contributor \& Team Leader}
\location{Jan 2013 – May 2013 | Palo Alto, CA \& Ithaca, NY}
\begin{tightemize}
\item Phabricator is used daily by Facebook, Dropbox, Quora, Asana and more.\item I created the Meme generator, the entire Lipsum application, ported Tokens to different apps, fixed many bugs and more in PHP and Shell.\item Led a team from MIT, Cornell, IC London and UHelsinki for the project.\end{tightemize}
\sectionsep
\end{comment}


%%%%%%%%%%%%%%%%%%%%%%%%%%%%%%%%%%%%%%
%    My EXPERIENCE
%%%%%%%%%%%%%%%%%%%%%%%%%%%%%%%%%%%%%%


%%%%%%%%%%%%%%%%%%%%%%%%%%%%%%%%%%%%%%
%     RESEARCH
%%%%%%%%%%%%%%%%%%%%%%%%%%%%%%%%%%%%%%

\section{Research Internships}
\runsubsection{University of Cambridge}
\descript{|Deep Reinforcement Learning \& Computational Social Science}
\location{ Jan 2020 | Cambridge, United Kingdom}
\begin{wrapfigure}[8]{l}{0.25\textwidth}
  \begin{center}
    \includegraphics[width=0.22\textwidth]{cam_logo.jpg}
  \end{center}
\end{wrapfigure}
Working on my undergraduate thesis under the guidance of Dr Shahar Avin and Dr Jess Whittlestone at the \href{https://www.cser.ac.uk/}{Centre for the Study of Existential Risk}. Here I developed a family of highly scalable and customizable agent-based models of AI research to understand the epistemology of machine learning research and how various factors like funding, research resources, hype and regulation impact it. These models were developed to have various deep reinforcement learning algorithms and game theoretic heuristics as a part of their decision making routines. Useful optimization techniques were developed to make them a useful computational tool for policy researchers, computer scientists, social scientists and philosophers.


\sectionsep


\runsubsection{MILA - Quebec AI Institute}
\descript{|Deep Reinforcement Learning }
\location{ May 2019 to Sep 2019| Montreal, Quebec, Canada}
\begin{wrapfigure}{l}{0.25\textwidth}
  \begin{center}
    \includegraphics[width=0.22\textwidth]{mila.png}
  \end{center}
\end{wrapfigure}

Worked with the Climate Change AI group at MILA : supervised by the Turing Laureate Prof Yoshua Bengio and Dr S. Karthik Mukkavilli. My work focused on building multi-agent deep reinforcement learning frameworks in modelling economies and their impact on climate change. I proposed a novel agent based integrated assessment model which uses deep reinforcement learning to better align the actions of economic agents to their impacts on the climate.This pioneering line of work combining agent-based modelling with reinforcement learning in application to policy discovery was appreciated by scholars from Harvard,Oxford ,Columbia, Waterloo and various other institutions.  

\sectionsep

%\includegraphics[scale=0.25]{caltechlogo.png}
\runsubsection{California Institute of Technology (Caltech)}
\descript{|Deep Learning for Biological imaging}
\location{May 2018 to Aug 2018|Pasadena,California,USA}
\begin{wrapfigure}{l}{0.25\textwidth}
  \begin{center}
    \includegraphics[width=0.22\textwidth]{caltechlogo.png}
  \end{center}
\end{wrapfigure}
Invited by Prof David Van Valen at Caltech for research in implementation of novel deep learning techniques in single cell imaging experiments. Our lab collaborated with Prof Marcus Convert's Lab at Stanford University for the collection of live microscopic cell imaging data.I worked on designing novel approaches for cell detection and cell segmentation .This included the modifications of existing algorithms for optimum results and I also proposed a segmentation neural network which outperforms the current state of the art for cell segmentation. My work contributed towards the development of Deepcell :enabling deep learning based biological image analysis in the cloud. 
\sectionsep

\runsubsection{Massachusetts Institute of Technology}
\descript{|Statistical Machine Learning }
\location{June 2017 to Aug 2017|Cambridge,Massachusetts,USA}
\begin{wrapfigure}{l}{0.25\textwidth}
  \begin{center}
    \includegraphics[width=0.20\textwidth]{mitlogo.png}
  \end{center}
\end{wrapfigure}
Invited by Prof. Gilbert Strang for research work in mathematical formulation of deep learning models - methods to formalize neural networks using statistical tools and tensor decomposition methods. We gave theoretical reasons for the unreasonable effectiveness of deep architectures from shallow ones. Was invited to write a chapter in his upcoming book.
\sectionsep
\end{minipage} 


\section{Invitations}

\runsubsection{Institute of Mathematical Sciences}
\descript{|Invited by director Prof Balasubramanian}
\location{19 to 21 January 2014 | Chennai, India}
Invited to discuss the progress made by him in q -series, prime counting functions and algorithms related to finding closed form expressions for integrals.Discussed novel methods in computing the number theoretic partition function and the methods in asymptotic  number theory as a whole. Also took a seminar titled " Computation of Some Integrals" for the Faculties of the Institute" \newline
\sectionsep


\runsubsection{Tata Institute of Fundamental Research }
\descript{| Invited by Professor Dipendra Prasad}
\location{8 to 10 Oct 2013 | Mumbai , India}
Invited to present and discuss my research work in Ramanujan mathematics , more specifically analytical number theory. Explored various properties on application of computer aided mathematical research with an emphasis on conjecture verification, automated mathematical proving along with pattern recognition.
\sectionsep



\section{Achievements}

\runsubsection{ICM 2014 -SEOUL }
\descript{| Youngest person to get abstract accepted}
\location{13 to 21 August 2014}
%\vspace{\topsep} % Hacky fix for awkward extra vertical space
\begin{tightemize}\item Abstract titled "New Families of Rogers Ramanujan Continued fractions" was accepted International Congress of Mathematics at age 15.
\end{tightemize}
\sectionsep

\runsubsection{Annual State Conferences}
\descript{| Invited Twice as a Speaker}
\location{14th Nov 2013 \& 25th Dec 2016}
%\vspace{\topsep} % Hacky fix for awkward extra vertical space
\begin{tightemize}\item Delivered a lecture on "Ramanujan Mathematics" and on "PI Matters" in 50th and 53th Annual  State Mathematics Conference respectively.
\end{tightemize}
\sectionsep

\runsubsection{Ted Talks}
\descript{| Invited twice as a Speaker}
\location{22 April 2017}
\begin{tightemize}
\item Delivered a Tedx talk on "Classrooms Beyond Boundaries" exploring themes of online education,the future and role of distributed computing projects.

\item Delivered a second Tedx talk on "The AI:Dilemma" exploring the past present and future of artificial intelligence with an emphasis on medicine and appealed the doctors to collaborate with computer scientist to shape a better future of medical AI.


\end{tightemize}
\sectionsep

\runsubsection{Erdos Number 2 }
\descript{| Youngest person to receive }
\begin{tightemize}
\item Collaborated with the Hungarian mathematician Mih´aly Bencze for "Asymptotic analysis on primes in a certain form" to receive Erdos number of 2.

\end{tightemize}
\sectionsep

\runsubsection{Computation of $\pi$ }
\descript{| Highest computation without GPU }
\begin{tightemize}
\item Computed $3.3$ trillion decimal digits of $\pi$ .
\end{tightemize}
\sectionsep



%%%%%%%%%%%%%%%%%%%%%%%%%%%%%%%%%%%%%%%%%%%%%%%%%%%%%%
\section{Industry Experience}

\runsubsection{Grok Video }
\descript{|Research Scientist }
\location{Feb 2020 - Present| Montreal, Quebec, Canada}

Developing highly scalable deep learning techniques for semi supervised semantic multimedia retrieval, incorporating ideas from Metric Learning, Weakly supervised Learning and Computer Vision. Devised a novel text-video-audio embedding which improves perfomance on down stream tasks by over $1\%$. Also handled the scalable deployment of these workflows using Kubernetes  and Apache Hadoop.
\sectionsep

\runsubsection{Taiyo LLC }
\descript{|Research Scientist }
\location{Feb 2020 - Present| Montreal, Quebec, Canada}
Focusing on the development of low latency stock trading pipelines leveraging state of the
art deep learning architectures and statistical techniques with Python, C++, AWS, Kubernetes and Docker.Also designing custom storage architectures, for achieving low latency and designing the framework for extensive backtesting of forecasting algorithms and strategies.
\sectionsep

\runsubsection{Sustlabs, IIT Bombay}
\descript{|Data Science Intern }
\location{Nov 2018 - Feb 2020| Mumbai,India}
Worked on the development of an IOT based smart metering device to collect electricity consumption data to over 100 times every second. Developed and implemented algorithms to improve the analysis and feedback from the billions of data points collected.
Developed and implemented novel temporal neural network architectures to perform near real-time  non-intrusive load monitoring.

\sectionsep
\vspace{2mm}

\runsubsection{Quiph}
\descript{|Natural Language Processing and Computer Vision Intern  }
\location{Aug 2017 - Oct 2018| Manglore,India}

Worked on a problem that involved tracking fast moving objects. Devised an algorithm based upon neural networks like architectures along with genetic algorithms and implemented it on the hardware using OpenCV and embedded C . It gave tracking efficiency of about 97\% . 
Another project was automatic language detection which I employed Recurrent Neural Networks and CNNs for an algorithm and implemented it with a system having over 200 languages and trained it with 108GB dataset it gives better than state of the art accuracy and I wrote a paper which is under review.





\sectionsep



\section{Miscellaneous}
\hline
\vspace{\topsep}
\runsubsection{Other Notable Accomplishments  }
\descript{|Scholarships,Honours,Felicitations }

\begin{tightemize}
\item Visiting Undergraduate Research Fellowship(VURP) by Caltech for pursuing a fully funded research internship at Caltech in the summer of 2018. 
\item Felicitated by The Income Tax Bar Association as “ Promising Gujarati"
\item Received the responsibility to develop \The only existing Ramanujan museum
in the world" which from the government of Tamil Nadu.
\item Given rare access to the original handwritten manuscripts of Srinivas Ramanujan to read , study and examine for 4 days during his visit at Indian Institute of Mathematical research.
\item The world record of highest number of decimal digits of $\pi$ calculation on a GPU uaided computer. 
\item Invited by IIT Bombay as a speaker in international conference on Python in Scientific Programming (Scipy 2017).
\end{tightemize}
\sectionsep


%%%%%%%%%%%%%%%%%%%%%%%%%%%%%%%%%%%%%%%%%%%%%%%



%%%%%%%%%%%%%%%%%%%%%%%%%%%%%%%%%%%%%%
%     AWARDS
%%%%%%%%%%%%%%%%%%%%%%%%%%%%%%%%%%%%%%


%%%%%%%%%%%%%%%%%%%%%%%%%%%%%%%%%%%%%%%%%%%%%%%%%%%

\section{Collaboarative Research Projects}
\hline
\vspace{\topsep}
\runsubsection{Digital Phenotyping for Mental Health }
\descript{|University of Montreal}
Collaborating with Prof Pierre Orban developing robust forecasting methods in digital phenotyping which can scale to large datasets and also have  performance robust to sparsity and data corruption. Specially developing techniques focusing on explainablity of the prediction and personalization to the individual patient. 
\sectionsep

\runsubsection{Deep Learning for Earth Sciences}
\descript{|Lawrence Berkeley National Laboratory}
Working with Dr S. Karthik Mukkavilli developing deep learning techniques suitable for applications in Earth Sciences. Including developement of PyDICEx - a robust, modular, extensible python package for precise simulation
of Integrated Assessment Models developed upon the DICE framework. Wrote highly efficient C++/Cython routines
for stochastic non-linear optimization. Currently working on an agent based analogue, on top of python library MESA and supports multi-RL based decision making routines and has an OPEN AI Gym like API for experimentation

\sectionsep

\runsubsection{Ultra Cryogenic Atom Analysis }
\descript{|MIT, Harvard University}
Developing Computer Vision techniques based on state of the art machine learning models , for the analysis, noise reduction and segmentation an of images of ultra-cold atoms from the Noble Prize winning experiments in the lab of Prof.  Wolfgang Ketterle , collaborating with Prof Martin Zwierlein at MIT-Harvard Center for Ultracold Atoms.

\sectionsep

\runsubsection{Burn Prognosis Using Deep Learning }
\descript{|Stanford University}
Working on the improvement of the current deep learning models in early burn diagnosis , using innovative architectures like use of fully connected layers and their variants on the BURNED dataset provided by Stanford University .  Also under the guidance of Orion Despo working for multiburn image analysis.
\sectionsep

\runsubsection{Global Climate data analysis }
\descript{|Berkeley Earth(Lawrence Berkeley National Laboratory) }
Implementing  data mining techniques -to mine the climate data from various parts of web . Along with this also Developing innovative analysis and comparison methods for interpretation and presentation of the data.Working with and under the guidance of Dr Robert Rhode.
\sectionsep

\runsubsection{Deep Learning powered Financial Mathematics }
\descript{|University of Toronto}
Working with team of  undergraduates in the Statistics department of University of Toronto , for the applications of deep learning algorithms in extremely dynamic and financial applications . Tested various methods in prediction of stocks , rates of currencies, cryptocurrencies currently focusing on Algo Trading with deep learning.
\sectionsep

\runsubsection{Polarimetric data analysis from RISAT-1 }
\descript{|ISRO (Indian Space Research Organization)}
Working on developing tools to store , process understand the polarimetric big data obtained by the satellite . Deploying various algorithms for the segmentation of the images converted from the data. Also implemented an improved version of Wishart Classifier which outperforms the state of the art methods of object and oil spill detection .
\sectionsep




\section{CS Projects} 
\hrule
\vspace{\topsep}
\begin{tabular}{lll}
DeepCell    & Contributor  &Implemented neural networks  Python Library of live segmentation segmentation \\
Golly	     & Co-creator & Open source platform for cellular automata aided with various rules.\\
Powder Toy	     & Developer  & A 2-D sandbox particle stimulator game written in Lua.\\

Gazebo	     & Programmer& Contributed in programing this massive open sourse robotics project.\\
Algoodo	     & Contributor& A 2-D physics stimulator \\
Blender     & Contributor  & Added efficient hair rendering algorithm  \\
SKData     & Contributor& Added to this library of Machine Learning and Statistics  \\
FANN & Contributor & Fast Artificial Neural Network Library which implements multilayer ANNs in C.\\
\end{tabular}
\sectionsep





%%%%%%%%%%%%%%%%%%%%%%%%%%%%%%%%%%%%%%
%     SOCIETIES
%%%%%%%%%%%%%%%%%%%%%%%%%%%%%%%%%%%%%%

\section{Mathematics and Computing Projects} 
\hline
\vspace{\topsep}
\begin{tabular}{ll}
Computation of $\pi$ & Computed 3.3 Trillion Decimal Digits- Discovered a quaternary converging recurrence algorithms.\\
 Ramanujan methods & Discovered multiple families of Ramanujan type hyper geometric series for $\pi$ and their generalization.\\
GIMPS&Involved in the discovery of last 2 Mersenne Primes through the GIMPS Project.\\
PrimeGrid& Discovered the currently the $2^{nd}$ largest repunit prime and several other types .\\
OEIS & Contributed over $4200$ sequences in the Online Encyclopedia of Integer Sequences.\\
Orbit@Home&Developing statistical models to identify the best places to identify near-Earth asteroids.\\
Mathematica&Developed over $10$ demonstrations on the Wolfram CDF Player coding in Mathematica .\\
Wikipedia & Contributed in over 2000 articles in Wikipedia, Wikia and Wiki Quotes. \\

\end{tabular}
\sectionsep


\section{Invited Talks}
\hline
\vspace{\topsep}
\begin{tabular}{ll}
Univerity of Oxford & Invited to the OII to deliver a talk on multi-agent RL in computational social sciences.\\
Stanford University & Invited to the department of computer science for a graphics talk about efficient 3D hair rendering. \\
Scipy India 2017 & Delivered a talk and conducted a workshop titled Next Generation Number theory and optimization .\\
Data Science Week & Conducted a 5 day workshop on Mathematical and Implementations aspects of Data Science. \\
St Xaviers College & Delivered a talk on "Chess Meets Mathematics" which is on game theory aided with Machine learning.\\
Techfest IEEE & Conducted a two days workshop covering the basics to the advanced level of Data science in Python.\\
St Xaviers College & Talk on  "Internet and Mathematics"on various internet aided tools for effective math research.\\
Tedx Nirma & "Classrooms beyond Boundaries" on the impact of online education and communities.\\
Tedx NHL & "The AI Dilemma" which explores relationships of upcoming AI Technologies with Medicine \\
\end{tabular}
$\cdots$ And in many other conferences , gatherings and meet-ups.
\sectionsep

\section{Books Written}
\hline
\vspace{\topsep}
\begin{itemize}
\item Glimpses of Ramanujan’s mathematics - Under Review in American Mathematical Association .
\item Mathematics of Nature and Nature of Mathematics -Writing it with Amarnath Krishnamurti.
\end{itemize}
\sectionsep


\section{Publications}
\hline
\vspace{\topsep}
\begin{itemize}
\item "Accelerating Unsupervised SAR Polarimetric Image Segmentation by Parallel Wishart Classifier" with Rohan Desai and
Pooja Shah ,GPU Technology Conference 2020 , San Jose,CA,USA

\item "LangDetectNet:Spoken Language Detection using parallel trainable Deep RCNN Architectures." GPU Technology Conference 2018 , San Jose,CA,USA.
\item "Next Generation Compressed Domain Video Hashing Using Deep Learning and Nvdia GPU" with Ekta Jayswal , GPU Technology conference 2018,San Jose,CA,USA.
\item "Computational Methods for Obtaining Unconditional Bounds for the Ramanujan’s Inequality on $\pi^2(x)$, P.C Vaidya National Conference of Mathematics 2018

\item “Disproof of a conjecture on d(N)” , Octagon Mathematical  Magazine  Volume 22. No. 1 April, 2014.
\item “AM-GM –HM Triples”, Octagon Mathematical Magazine. -Vol 2 No.2  Oct 2014 .
\item "S Union P is negated by F then d --  Octagon Mathematical Magazine. -Vol 2 No.2  Oct 2014 .
\item  “Classification of number theoretic sequences involving primes and special functions “ published in  Octagon Mathematical Magazine. -Vol 2 No.2  Oct 2014 
\item Does there exits proof with S* "
\item  “Closed forms of some logarithmic integrals with irrational exponents ”  published in Octagon Mathematical Magazine.- 
\item “Asymptotic analysis on primes in a certain form - as conjectured by M. Bencze” with M. Bencze  published in Octagon Mathematical Magazine.
\item “Exact formula for the number of primes in the form 4n+3 ≤ x in terms of the Riemann R function”published in Octagon Mathematical Magazine-
\item "Representation Non Perfect Squares by Triangular numbers" published in Octagon Mathematical Magazine.- Oct 2014
\item Legal:Illegal Chess Games"  published in Octagon Mathematical Magazine.- Oct 2014
\item 
"Impossibility of construction of Fibonacci and Lucas Magic Squares (OQ. 4661)" published in Octagon Mathematical Magazine.- Oct 2014
\item Application divergent Mellin transform in evaluation of series and limits of sequences”  published in “The Indian Journal of Pure and Applied Mathematics
\item Developed software for the article "Maxillary and Mandibular Arch Perimeter Prediction Using Ramanujan's Equation for the Ellipse-In vitro Study " published in  British Journal of Medicine & Medical Research- Sep 2016.
\end{itemize}
\sectionsep

\end{document}